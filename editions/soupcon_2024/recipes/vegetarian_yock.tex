\recipe{Vegetarian Yock}
{Ian Calloway}
{Yock (or yock-a-mein) is a noodle soup from the Tidewater region of Virginia, the southeast of the state along the Chesapeake Bay and the Atlantic Ocean, where I grew up. Its origins trace back to the early 1900s, when Chinese immigrants first brought their culinary traditions to Virginia. Due to segregation, they established restaurants in black neighborhoods, where this noodle soup evolved to reflect the local tastes and culture of the black community of the Tidewater. Nowadays, it's common to see yock on a Chinese food menu in the Tidewater area as well as the occasional yock fundraiser in black churches. I haven't seen this dish yet in Michigan, so this will be my first attempt cooking it outside of the 757. While I don't have a family recipe to work from, this attempt is cobbled together from a few sources and adapted to be vegetarian.
}
{ % Ingredients Command
    \ingredients{
\item  20 oz vegan chicken						
\item 8 oz lo mein noodles	
\item 4 cups vegetable broth		
\item 3 tablespoons soy sauce	
\item 1 tablespoon vegan Worcestershire sauce
\item 1 teaspoon garlic powder
\item 1 teaspoon onion powder
\item 1/2 teaspoon sugar (to balance flavors)
\item Salt and black pepper to taste
}
}
{ % Instructions Command
    \instructions{
\item Cook the noodles according to package instructions. Drain and set aside.
\item In a large pot, combine the broth, soy sauce, Worcestershire sauce, garlic powder, onion powder, and sugar. Bring to a simmer and taste for seasoning. Adjust salt, pepper, or soy sauce as needed.
\item Pan fry the vegan chicken and add to the broth.
\item Ladle the hot broth and protein mixture over the noodles.
\item  Garnish each bowl with halved boiled eggs, green onions, a drizzle of hot sauce, or a splash of vinegar.
  }
}
{}
{vegetarian_yock.jpg}
{American Regional Delicacies}
