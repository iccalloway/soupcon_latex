\recipe{Chicken Noodle}
{Jonathan Earley}
{Before heading up to Leelanau Peninsula, I prepared a large pot of chicken noodle soup, made in the style my dad would, with special homemade noodles. My wife, Sky, and I arrived at the cabin, nestled in the middle of a wide winter prairie, the snow blanketing the ground. I reheated the soup, eager to settle into the quiet of the cabin. But in the rush to unload the car, I accidentally locked us both out. I was in my slippers, with no phone to call for help. After a bit of panic walking up and down a rural road, I flagged down a passing driver who helped me contact the host, who gave me a passcode to get back inside. Thankfully, the soup was just fine, and the noodles were perfect. Sky and I couldn’t help but laugh at the absurdity of it all as we finally sat down to enjoy the comforting meal. Now, I’m excited to share this soup with you.}
{ % Ingredients Command
    \ingredients{
    
\item 4 bone in, skin off split chicken breasts
\item 10-14 cups of water
\item 8 stalks of celery
\item 4 carrots
\item 6 cloves garlic
\item 2 yellow onion
\item 1 bay leaf
\item 2 tsp. black peppercorns
\item 1 tbsp. salt
\item 2 tbsp. olive oil

\vspace{0.5em}\underline{Egg noodle ingredients}
\item 2 eggs
\item 1 tsp. salt
\item 1 tsp. fresh cracked pepper
\item 4 tbsp. milk
\item 2 cups flour

}
    

}
{ % Instructions Command
    \instructions{
\item Prepare the broth -- Place the bone-in chicken breasts in the largest pot you got. Pour in the cups of water, ensuring the chicken is fully submerged. Consider adding more water because you might regret not having enough broth. Gently put your chicken to rest and ensure they're covered. Add the bay leaf, 4 celery stalks, 2 carrots (halved), 1 yellow onion (halved), 6 garlic cloves (smash them), peppercorns, and 1 tbsp. of kosher salt. These ingredients are for flavoring the broth, so there’s no need to dice them—just add them whole as they are.
\item Simmer the broth -- Bring the pot to a boil over medium-high heat, then reduce the heat to maintain a gentle simmer. Let the broth simmer for about 2 hours.
\item Prepare the egg noodles -- In a mixing bowl, combine eggs, milk, salt, and pepper. Stir in one cup of flour until smooth. Add additional flour, just a small spoonful at a time, until the dough comes together in a ball, but is still slightly sticky. Roll dough out on a floured surface until less than 1/4 inch thick and let it rest for at least 30 minutes. Using a pizza cutter, cut the noodles into long strips as you like. They can rest and dry out until we need them.
\item Strain the Broth -- Once the chicken is cooked and tender, use tongs to carefully remove the chicken from the pot and transfer it to a large cutting board. Let the chicken cool enough to handle.
\item While the chicken cools, place a fine-mesh strainer over a separate large bowl or pot. Carefully pour the broth through the strainer to remove the vegetables, bay leaf, and any other solids. Discard the solids and set the clean broth aside.
\item Shred the chicken -- Once the chicken is cool enough to handle, remove the bones and discard them. Carefully pull the meat from the bones, making sure to remove all small bones. Set the shredded chicken aside. I like to cut the chicken into smaller soup spoon sized pieces, but you do what you like.
\item Prepare the vegetables -- Dice the remaining 1 onion, 4 celery stalks, and 2 large carrots.
\item Sauté the vegetables -- In the same pot, heat over medium-high heat, add 2 tablespoons or enough to cover the bottom of the pot. Add the diced vegetables, seasoning with a pinch of salt and pepper. Sauté for about 8 minutes, or until the vegetables are tender.
\item Combine broth and chicken -- Pour the strained broth back into the pot with the sautéed vegetables. Add the shredded chicken to the pot—add as much or as little as you like.
\item Add egg noodles -- Bring soup to a high simmer and add the noodles and stir for about 10 minutes. Turn off the heat. Let the soup rest and cool for about 15 more minutes.
\item Serve, and enjoy

    }
}
{}
{}
{Meaty Broths}
{}
