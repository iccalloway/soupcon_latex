\recipe{Yoğurt Çorbası}
{Demet Kayabaşı}
{Yoğurt Çorbası (yogurt soup), also known as Yayla Çorbası (highland soup), though I don't use that name, is one of the staples of Turkish home-cooking that I grew up with. There are many recipes and variations of it within Turkey, and in neighboring or nearby countries to Turkey, such as Iran, Iraq, Armenia, and likely Lebanon too, etc., especially in terms of what herbs it includes. I make it as I've seen from my parents and grandparents, and as I've been cooking it for years, using dried spearmint leaves. Other recipes/variations can include parsley or purslane. Likewise, the version I cook has rice, but other variations might use chickpeas or both chickpeas and rice. I chose this soup because its main ingredient is yogurt, which is quite stereotypically Turkish to use in any kind of dish. And I think it's a good representation of what I consider a culturally accurate comfort meal from Turkish cuisine. In addition, it's likely something novel for the rest of the SoupCon attendees. Like other soups in Turkish cuisine, it can be cooked and eaten for any course' including breakfast (more commonly so in rural areas), but it is typically served at lunch or dinner before the main course or can be a meal by itself when accompanied by bread or croutons.}
{ % Ingredients Command
    \ingredients{
\item []The base
\item Water (or chicken stock) - a little under 2 liters, like 7-7½  cups?
\item White rice - any variety. eyeball it depending on how ricey you want it lol but I’d say start with ¼ cups and see if you feel like adding more once it’s in the pot. Maybe don’t add more than ½ cups.\\
\item[] The thickening agent 
\item Flour - 1 tbsp
\item One egg yolk
\item Yogurt  - 1 cup, plain yogurt, preferably with %2% or %5% (or more) milk fats.
\item Dry spearmint leaves  - idk how much, don’t be shy with it. Maybe start with ¼ cups and gradually add more. It’s really supposed to be the star of the show.\\
\item[] The garnish 
\item Eyeball this section entirely, my recommended butter to mint ratio would be 2-1. I used a stick of butter for the x3 recipe, so 1/3 stick should be a good place to start for these measurements.
\item Butter - I’ve seen this made with olive oil, too. It’ll change the taste, and butter goes better with it if you use chicken stock, but olive oil is just fine if you’re using water.
\item Dry spearmint leaves\\
\item[] Spices 
\item Salt and pepper to taste
\item Red chili flakes go well with it if you like it. I think using chilli oil overpowers the mint though, so I wouldn’t use it.
    }
}
{ % Instructions Command
    \instructions{
\item[] The base
\item Add the water or the chicken stock to your pot, and put it on high heat until it starts boiling.
\item Add the rice once your base is boiling. Keep it on medium or medium-high until the rice is really soft and mushy, and doesn’t look like rice anymore. Idk how long it takes, I check on it in 15-20 minutes usually? Maybe a half an hour, but not more than that.\\
\item[] The thickening agent
\item Combine the ingredients of the thickening agent in a separate bowl in no specific order.
\item Add some of the base to the thickening agent gradually. One ladle should be enough, but if you’re using a spoon, try 5-6 tablespoons at least. Mix it well until the mixture is smooth.\\
\item[] Combining the two  
\item Bring the heat down to low or medium-low, and bring it to a simmer.
\item Add the thickening agent to the water/stock slowly while mixing constantly. 
\item Keep mixing it for a few minutes at least until you’re sure that all the mixture is perfectly blended and it’s not curdled. The rice pieces might make it hard to assess if it’s curdled or not. In that case, just taste one of the bits. It’s okay if it tastes like rice. It’s curdled if it tastes really sour (a floury yogurt bit).
\item If it did end up curdled, you can strain it or blend it with an immersion blender. It’ll change the texture a bit, but it’ll most likely save the soup, at least if it hasn’t soured a lot. If it soured a lot, you can probably still save it by adding more of the garnish because the butter will help balance the acidity out, and the mint will help mask the sourness.
\item Once the thickening agent seems to be well incorporated, add in the dry mint (A LOT)  and bring everything to a simmer. Let it simmer for about 10 minutes.
\item Add salt and pepper to taste once it’s done simmering. You can take it away from the heat at this point.\\
\item[] The garnish
\item For the garnish, melt the butter in a saucepan and add dry mint to it. Make sure to add it after the butter is already melted so the mint doesn’t burn. Once it starts sizzling, take it off the heat and drizzle it over the soup. You can do this to the whole pot and give it a mix, or drizzle some on each individual serving.
}
}
{
DO NOT add salt before the soup is done cooking, otherwise the yogurt in the thickening agent might get curdled.\\\\
If you have fresh mint, dry it first.\\\\
The end result should be more liquidthan puree-like. If you end up with a puree-like consistency, you either need to start with more water next time, or cut back on rice and/or the amount of the thickening agent. Alternatively, you might have forgotten about it simmering on the stove, and it evaporated way more than it should. You can dilute it by adding some boiling water and having it simmer for a while if you catch it before you add the garnish.

}
{yogurt_cobasi.jpg}
{Mediterranean}
{Yogurt Soup}
