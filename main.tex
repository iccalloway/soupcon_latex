\documentclass{anthology}
\begin{document}

\tableofcontents

\recipeCategory{Mediterranean}
\recipe{Avgolemono (Greek Egg and Lemon Chicken Rice Soup)}
{Mara Bollard \& Nate Smith}
{A Bollard family favourite! Passed down from my mother’s Greek-Cypriot family and my and my siblings’ \#1 comfort food. Fortunately Nate likes it too (and, bless him, he now is in charge of lemon squeezing).}
{ % Ingredients Command
    \ingredients{
        \item 2-3 lbs of bone-in chicken pieces, preferably dark meat (I like using whole legs but whatever combo of pieces you have works! You can even use a whole chicken if you really want to go for it)
	\item 4-5 carrots (peeled or unpeeled, it’s up to you)
	\item ½ bunch of celery
	\item Chicken stock powder or paste, to taste
	\item 1.5 cups of medium or long-grain white rice, unrinsed (the shorter the grain, the starchier and creamier the soup will be; long-grain rice will give a more brothy consistency - both are great, follow your heart.)
	\item 5-6 lemons, juiced (enough for 1.5-2 cups of lemon juice - this amount leads to a very lemony final product, which is my family’s jam, but feel free to adjust if this lemon-level is too aggressive)
	\item 5-6 eggs
	\item Salt and pepper, to taste

    }
}
{ % Instructions Command
    \instructions{
	\item Roughly chop your celery (leaves included, if they’re still on there) and carrots into 2-inch-ish pieces. Add vegetables to a large soup pot along with the chicken/chicken pieces. Fill the pot with cold water, add plenty of salt, and simmer over low-medium heat until everything is cooked through and tender. This could be anywhere from 1.5-3 hours, depending on how big your pot is, how much of a rush you’re in, and how high your heat is. If needed, skim the top periodically with a slotted spoon. 

	\item Remove the chicken from the pot and set aside. Taste the broth. If you think the chicken flavor needs amping up, add some stock powder/paste (I like to add a generous spoonful of the roasted chicken Better than Bouillon paste at this point) and top up with some more water if the liquid level has gone down more than you’d like. 

	\item Bring broth back up to a slow simmer, then stir in the rice. Simmer gently until rice is cooked through. Meanwhile, shred the cooked chicken into bite-sized pieces (discarding skin/bones) and juice your lemons. When rice is done, turn off the heat and let cool for a few minutes. 

	\item Break eggs into a large mixing bowl and whisk until frothy. Whisk in the lemon juice. 

	\item Place your mixing bowl right next to the soup pot. While constantly whisking the egg/lemon mixture, very slowly ladle a small amount of the warm soup broth into the egg/lemon mixture. Gradually drizzle in more of the broth, whisking the whole time, until you’ve added at least a few cups of broth to the mixing bowl. (Take your time during this part - if you add the hot broth to the egg mixture too quickly, you risk curdling the eggs - and enjoy those whisking forearm strength gainz.)  

	\item Pour the (now tempered!) egg/lemon mixture and the shredded chicken back into the pot and stir to combine. Taste one more time, add salt/pepper as desired, then serve and enjoy! 

    }
}
{}
{images/avgolemono.jpeg}

\recipeCategory{Meaty Broths}
\recipe{Caldo Verde}
{Lauren Bloom}
{This soup came from an old issue of a cooking magazine. It didn’t arrive with any nostalgia, but has picked some up over the years as it’s been made annually for a gals weekend in the woods.}
{ % Ingredients Command
    \ingredients{
 \item ¼ cup extra-virgin olive oil
\item 12 ounces Spanish-style chorizo sausage, cut into 1/2-inch pieces
\item 1 onion, chopped fine
\item 4 garlic cloves, minced
\item Salt and pepper
\item ¼ teaspoon red pepper flakes
\item 2 pounds Yukon Gold potatoes, peeled and cut into 3/4-inch pieces
\item 4 cups chicken broth
\item 4 cups water
\item 1 pound collard greens, stemmed and cut into 1-inch pieces
\item 2 teaspoons white wine vinegar
    }
}
{ % Instructions Command
    \instructions{
\item Heat 1 tablespoon oil in Dutch oven over medium-high heat until shimmering. Add chorizo and cook, stirring occasionally, until lightly browned, 4 to 5 minutes. Transfer chorizo to bowl and set aside. Reduce heat to medium and add onion, garlic, 1 1/4 teaspoons salt, and pepper flakes and season with pepper to taste. Cook, stirring frequently, until onion is translucent, 2 to 3 minutes. Add potatoes, broth, and water; increase heat to high and bring to boil. Reduce heat to medium-low and simmer, uncovered, until potatoes are just tender, 8 to 10 minutes.
\item Transfer 3/4 cup solids and 3/4 cup broth to blender jar. Add collard greens to pot and simmer for 10 minutes. Stir in chorizo and continue to simmer until greens are tender, 8 to 10 minutes longer.

\item Add remaining 3 tablespoons oil to soup in blender and process until very smooth and homogeneous, about 1 minute. Remove pot from heat and stir pureed soup mixture and vinegar into soup. Season with salt and pepper to taste, and serve. (Soup can be refrigerated for up to 2 days.)

    }
}
{}
{images/caldo_verde.jpg}




\end{document}